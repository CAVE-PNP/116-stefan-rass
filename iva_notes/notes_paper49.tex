\documentclass[12pt, xcolor=dvipsnames]{scrartcl}
\usepackage[utf8]{inputenc}
\usepackage{mathptmx}
\usepackage[T1]{fontenc}
\usepackage{lmodern}
\usepackage{graphicx}
\usepackage[english, naustrian]{babel}
\usepackage[draft=false, kerning=true]{microtype} 
\usepackage{amsmath, amssymb, amstext, amsthm}
\usepackage{hyperref}
\usepackage[a4paper,left=2.6cm, right=2.9cm,top=3.5cm, bottom=3.5cm]{geometry}
\addtokomafont{sectioning}{\rmfamily} %Titelzeilen
\theoremstyle{definition}
\newtheorem{definition}{Definition}%[section]
\newtheorem*{formulierung}{Formulierung}%[section]
\newtheorem*{theorem}{Satz}
\usepackage{physics}
\usepackage{mathtools}
\usepackage[dvipsnames]{xcolor}
\usepackage{color}
\usepackage{tikz}
%\usepackage{pgf-pie}
\usetikzlibrary{arrows.meta}
\usetikzlibrary{calc}
\usepackage{verbatim}
\usetikzlibrary{arrows,shapes}
\usepackage{graphics}
\usepackage{parskip}
\usepackage[onehalfspacing]{setspace} % 1.5 Zeilenabstand
\usepackage[justification=centering]{caption}
\usepackage{enumitem}
\usepackage[sort&compress,numbers]{natbib}
\usepackage{caption}
\usepackage{subcaption}
\usepackage{booktabs, makecell}
\usepackage{float}
\usepackage{listings}
\usepackage{colortbl} 
\usepackage{xfrac}
\usepackage{array}
\usepackage{pgfplots}
\usepackage{ulem}
\usepackage{listings}
\usepackage{romannum}
\usepackage{xcolor}
\usepackage{multicol}
\usepackage{calc}
\usetikzlibrary{positioning,matrix, arrows.meta}

\tikzset{
    table nodes/.style={
        draw,
        align=center,
        minimum height=7mm,
        minimum width =7mm,
        text depth=0.5ex,
        text height=2ex
    },      
    table/.style={
        matrix of nodes,
        row sep=-\pgflinewidth,
        column sep=-\pgflinewidth,
        nodes={
            table nodes
          },
        nodes in empty cells
     }
}


\definecolor{codegreen}{rgb}{0,0.6,0}
\definecolor{codegray}{rgb}{0.5,0.5,0.5}
\definecolor{codepurple}{rgb}{0.58,0,0.82}
\definecolor{mygreen}{RGB}{28,172,0} % color values Red, Green, Blue
\definecolor{mylilas}{RGB}{170,55,241}
\definecolor{backcolour}{rgb}{0.95,0.95,0.92}

\theoremstyle{definition}
%\newtheorem{definition}{Definition}

\title{P/NP Notizen}
\subtitle{Paper \#49}
\date{\today}

\begin{document}
\maketitle

\newpage

\section*{Aufbau von Paper \#49}
Titel: \textit{A polynomial-Time Algorithm for the Maximum Clique Problem} \\
Author: \textit{Zohreh O. Akbari} \\

\subsection*{\Romannum{1}. Introduction}
Eine wichtige Konsequenz des Cook-Levin Theorems (SAT Problem ist $NP$-vollst"andig) ist, dass sobald ein Problem aus $NP$ in polynomieller Zeit l"osbar ist, alle Probleme aus $NP$ in polynomieller Zeit l"osbar sind. Das w"urde bedeuten, dass $P = NP$.

Paper \#49 pr"asentiert einen polynomiellen Algorithmus f"ur das \textit{Maximum Clique Problem}. Der Autor folgert also $P = NP$.

\subsection*{\Romannum{2}. The Maximum Clique Problem}

\begin{definition}[Graph]\ \\
    Sei $V$ eine Knotenmenge und $E \subseteq \{ \{i,j\} \mid i,j \in V, i \neq j\}$. Dann hei"st das Paar $G = (V,E)$ ein Graph.
\end{definition}

\begin{definition}[Vollst"andiger Graph]\ \\
    Ein Graph $G = (V,E)$ hei"st vollst"andig, wenn alle Knoten paarweise adjazent sind. Also $\forall ~ i,j \in V,~ i \neq j \Longrightarrow (i,j) \in E$. Falls jeder Knoten Knotengrad $|V|-1$ besitzt, so ist $G$ vollständig.\\
    Autor bezeichnet mit $\varphi$ (vermutlich) den vollst"andigen Graphen.
\end{definition}

\begin{definition}[Clique]\ \\
    Sei $C \subseteq V$. $C$ nennt man Clique, falls $G = (C,E')$ vollst"andig ist.
\end{definition}

\begin{definition}[Gr"o"ste Clique]\ \\
    $\omega (G)$ ist die gr"o"ste Clique in $G$. Also 
    \[ \omega (G) = \max\{|S| : S \text{ ist eine Clique in }G\} \]
\end{definition}

\begin{definition}[Gr"o"ster Teilgraph von $G$ der $\alpha$ enth"alt]\ \\
    Sei $\alpha$ jener Knoten mit geringstem Knotengrad. 
    \[ \alpha = \{j \in V : deg(j) \text{ ist minimal}\} \]
    Den Teilgraph, der $\alpha$ und alle inzidenten Knoten zu $\alpha$ enh"alt, bezeichnet der Autor als gr"o"sten Teilgraph in $G$. \\ \textit{($\alpha$ und alle seine Nachbarn samt deren Kanten)}
\end{definition}

\subsection*{\Romannum{3}. A Polynomial-Time Algorithm for the Maximum Clique Problem}

Pseudocode:

\lstdefinestyle{mystyle}{
    backgroundcolor=\color{backcolour},   
    commentstyle=\color{codegreen},
    keywordstyle=\color{blue},
    numberstyle=\tiny\color{codegray},
    stringstyle=\color{codepurple},
    basicstyle=\ttfamily\footnotesize,
    breakatwhitespace=false,         
    breaklines=true,                 
    captionpos=b,                    
    keepspaces=true,                 
    numbers=left,                    
    numbersep=5pt,                  
    showspaces=false,                
    showstringspaces=false,
    showtabs=false,                  
    tabsize=2,
    aboveskip=\medskipamount,
    moredelim=**[is][\color{blue}]{@}{@},
    moredelim=**[is][\color{codepurple}]{€}{€}
}

\lstset{style=mystyle}

\begin{lstlisting}[escapeinside={(*}{*)}]
€MaxClique€(G) {
    @if@ (G is a complete graph)
        @for each@ vertex of G: v
            @if@ (|V| - 1 > max C[v])
                max C[v] := |V|;
                make max CP[v] point to a linked list containing V;
    @else@
    find the vertex of lowest degree: (*$\alpha$*)
    find the largest subgraph of G in which (*$\alpha$*) exists: G'(V',E')
    €MaxClique€(G');
    @if@ (V - (*$\alpha \neq \varphi$*))
        €MaxClique€(G - (*$\alpha$*));
}
\end{lstlisting}

\newpage

Beispiel: 

          \begin{figure}[H]
            \centering
          \begin{tikzpicture}
            \draw (2.5,4) node[draw, circle, inner sep=1.5pt]               (a) {a};
            \draw (4,4) node[draw, circle, inner sep=1.5pt]                   (b) {b};
            \draw (1,2) node[draw, circle, inner sep=1.5pt]                  (c) {c};
            \draw (3,1) node[draw, circle, inner sep=1.5pt]                  (d) {d};
            \draw (5,1) node[draw, circle, inner sep=1.5pt]                  (e) {e};
            \draw (7,0.5) node[draw, circle, inner sep=1.5pt]                  (f) {f};
            \draw (0,0) node[draw, circle, inner sep=1.5pt]                    (g) {g};
            \draw (2,0) node[draw, circle, inner sep=1.5pt]                  (h) {h};
            \draw (4,0) node[draw, circle, inner sep=2pt]                  (i) {i};
            \draw (g) -- (c);
            \draw (c) -- (a);
            \draw (c) -- (e);
            \draw (c) -- (d);
            \draw (c) -- (h);
            \draw (h) -- (d);
            \draw (h) -- (a);
            \draw (h) -- (i);
            \draw (i) -- (a);
            \draw (i) -- (b);
            \draw (i) -- (e);
            \draw (i) -- (f);
            \draw (f) -- (e);  
            \draw (f) -- (b);
            \draw (e) -- (b);
            \draw (e) -- (a);
            \draw (d) -- (e);
            \draw (d) -- (a);
            \draw (d) -- (b);
            \draw (h) -- (e);

            \node[text width=2cm] at (3.5,-1) 
              {Graph $G$};

            \end{tikzpicture}
          \end{figure}

          Ausganglage ist der obige beliebige Graph $G$. \\ 
          \texttt{\textcolor{codepurple}{1},7: }$G$ ist nicht vollständig, da nicht jeder Knoten Knotengrad $|V|-1$ besitzt. \\
          \texttt{8-9: }Knoten mit minimalen Knotengrad ist $g$, also $\alpha := g$. \\

        \begin{figure}[h]
            \begin{minipage}[c]{6cm}
                $G' = (\{c,g\},\{\{c,g\}\})$ 
            \end{minipage}%
            \begin{minipage}[c]{\textwidth-7cm}
                \begin{tikzpicture}
                    \draw (0.5,1) node[draw, circle, inner sep=1.5pt]                  (c) {c};
                    \draw (0,0) node[draw, circle, inner sep=1.5pt]                    (g) {g};
                    \draw (g) -- (c);
                    \end{tikzpicture}
            \end{minipage}
         \end{figure}

         \texttt{10,2-6: }$G'$ ist vollständig. \\
         \begin{tikzpicture}
            \matrix[table] (A)
            { |[draw=none]|
                  &a&b&c&d&e&f&g&h&i\\
              maxC &  &  & 2 &  &  &  & 2 &  &  \\
              maxC & & & & & & & & & \\
            };
            \matrix[table,below=of A-3-2] (B) {c\\g\\};
            %\matrix[table,below=of A-3-6] (C) {b\\e\\f\\i\\};
            %\matrix[table,below=of A-3-8] (D) {a\\c\\d\\e\\};
            \foreach \s/\t in {8/B,4/B}%,3/C,7/C,10/C,2/D,4/D,5/D,6/D,9/D}
              \draw[-stealth',shorten >=3pt,shorten <=3pt] (A-3-\s.south) -- (\t-1-1.north);
          \end{tikzpicture}

          \texttt{11: }$V - g$ ist nicht vollständig. \\

          \begin{figure}[H]
            \centering
          \begin{tikzpicture}
            \draw (2.5,4) node[draw, circle, inner sep=1.5pt]               (a) {a};
            \draw (4,4) node[draw, circle, inner sep=1.5pt]                   (b) {b};
            \draw (1,2) node[draw, circle, inner sep=1.5pt]                  (c) {c};
            \draw (3,1) node[draw, circle, inner sep=1.5pt]                  (d) {d};
            \draw (5,1) node[draw, circle, inner sep=1.5pt]                  (e) {e};
            \draw (7,0.5) node[draw, circle, inner sep=1.5pt]                  (f) {f};
            \draw (2,0) node[draw, circle, inner sep=1.5pt]                  (h) {h};
            \draw (4,0) node[draw, circle, inner sep=2pt]                  (i) {i};
            \draw (c) -- (a);
            \draw (c) -- (e);
            \draw (c) -- (d);
            \draw (c) -- (h);
            \draw (h) -- (d);
            \draw (h) -- (a);
            \draw (h) -- (i);
            \draw (i) -- (a);
            \draw (i) -- (b);
            \draw (i) -- (e);
            \draw (i) -- (f);
            \draw (f) -- (e);  
            \draw (f) -- (b);
            \draw (e) -- (b);
            \draw (e) -- (a);
            \draw (d) -- (e);
            \draw (d) -- (a);
            \draw (d) -- (b);
            \draw (h) -- (e);

            \node[text width=2cm] at (3.8,-1) 
              {$V - g$};

            \end{tikzpicture}
          \end{figure}

          \texttt{12,8,9: }Knoten mit minimalen Knotengrad ist $f$, also $\alpha := f$. \\
          \newpage
          $G' = (\{b,e,f,i\},\{\{b,e\},\{b,f\},\{b,i\},\{e,f\},\{e,i\},\{f,i\}\})$ 
          \begin{figure}[h]
          \begin{tikzpicture}
            \draw (4,4) node[draw, circle, inner sep=1.5pt]                   (b) {b};
            \draw (5,1) node[draw, circle, inner sep=1.5pt]                  (e) {e};
            \draw (7,0.5) node[draw, circle, inner sep=1.5pt]                  (f) {f};
            \draw (4,0) node[draw, circle, inner sep=2pt]                  (i) {i};
            \draw (i) -- (b);
            \draw (i) -- (e);
            \draw (i) -- (f);
            \draw (f) -- (e);  
            \draw (f) -- (b);
            \draw (e) -- (b);

            \end{tikzpicture}
          \end{figure}

          \texttt{10,2-6: }$G'$ ist vollständig. \\
         \begin{tikzpicture}
            \matrix[table] (A)
            { |[draw=none]|
                  &a&b&c&d&e&f&g&h&i\\
              maxC &  & 4 & 2 &  & 4 & 4 & 2 &  & 4 \\
              maxC & & & & & & & & & \\
            };
            \matrix[table,below=of A-3-2] (B) {c\\g\\};
            \matrix[table,below=of A-3-6] (C) {b\\e\\f\\i\\};
            %\matrix[table,below=of A-3-8] (D) {a\\c\\d\\e\\};
            \foreach \s/\t in {8/B,4/B,3/C,6/C,7/C,10/C}%,3/C,7/C,10/C,2/D,4/D,5/D,6/D,9/D}
              \draw[-stealth',shorten >=3pt,shorten <=3pt] (A-3-\s.south) -- (\t-1-1.north);
          \end{tikzpicture}

          \texttt{11: }$V - f$ ist nicht vollständig. \\

          \begin{figure}[H]
            \centering
          \begin{tikzpicture}
            \draw (2.5,4) node[draw, circle, inner sep=1.5pt]               (a) {a};
            \draw (4,4) node[draw, circle, inner sep=1.5pt]                   (b) {b};
            \draw (1,2) node[draw, circle, inner sep=1.5pt]                  (c) {c};
            \draw (3,1) node[draw, circle, inner sep=1.5pt]                  (d) {d};
            \draw (5,1) node[draw, circle, inner sep=1.5pt]                  (e) {e};
            \draw (2,0) node[draw, circle, inner sep=1.5pt]                  (h) {h};
            \draw (4,0) node[draw, circle, inner sep=2pt]                  (i) {i};
            \draw (c) -- (a);
            \draw (c) -- (e);
            \draw (c) -- (d);
            \draw (c) -- (h);
            \draw (h) -- (d);
            \draw (h) -- (a);
            \draw (h) -- (i);
            \draw (i) -- (a);
            \draw (i) -- (b);
            \draw (i) -- (e);
            \draw (e) -- (b);
            \draw (e) -- (a);
            \draw (d) -- (e);
            \draw (d) -- (a);
            \draw (d) -- (b);
            \draw (h) -- (e);

            \node[text width=2cm] at (3.8,-1) 
              {$V - f$};

            \end{tikzpicture}
          \end{figure}

          \texttt{12,8,9: }Knoten mit minimalen Knotengrad ist $b$, also $\alpha := b$. \\
          $G' = (\{b,d,e,i\},\{\{b,e\},\{b,d\},\{b,i\},\{d,e\},\{e,i\}\})$ 
          \begin{figure}[h]
          \begin{tikzpicture}
            \draw (4,4) node[draw, circle, inner sep=1.5pt]                   (b) {b};
            \draw (5,1) node[draw, circle, inner sep=1.5pt]                  (e) {e};
            \draw (3,1) node[draw, circle, inner sep=1.5pt]                  (d) {d};
            \draw (4,0) node[draw, circle, inner sep=2pt]                  (i) {i};
            \draw (i) -- (b);
            \draw (i) -- (e);
            \draw (d) -- (e);  
            \draw (d) -- (b);
            \draw (e) -- (b);

            \end{tikzpicture}
          \end{figure}

          \texttt{10: }$G'$ ist nicht vollständig. \\
          \texttt{8-9: }Knoten mit minimalen Knotengrad ist $d$ oder $i$, also o.B.d.A? $\alpha := d$. \\
          $G' = (\{b,d,e\},\{\{b,e\},\{b,d\},\{d,e\}\})$ 

          \begin{figure}[h]
          \begin{tikzpicture}
            \draw (4,4) node[draw, circle, inner sep=1.5pt]                   (b) {b};
            \draw (5,1) node[draw, circle, inner sep=1.5pt]                  (e) {e};
            \draw (3,1) node[draw, circle, inner sep=1.5pt]                  (d) {d};
            \draw (d) -- (e);  
            \draw (d) -- (b);
            \draw (e) -- (b);

            \end{tikzpicture}
          \end{figure}


          \texttt{10,2-6: }$G'$ ist vollständig. \\
         \begin{tikzpicture}
            \matrix[table] (A)
            { |[draw=none]|
                  &a&b&c&d&e&f&g&h&i\\
              maxC &  & 4 & 2 & 3 & 4 & 4 & 2 &  & 4 \\
              maxC & & & & & & & & & \\
            };
            \matrix[table,below=of A-3-2] (B) {c\\g\\};
            \matrix[table,below=of A-3-6] (C) {b\\e\\f\\i\\};
            \matrix[table,below=of A-3-4] (D) {b\\d\\e\\};
            \foreach \s/\t in {8/B,4/B,3/C,6/C,7/C,10/C,5/D}%,3/C,7/C,10/C,2/D,4/D,5/D,6/D,9/D}
              \draw[-stealth',shorten >=3pt,shorten <=3pt] (A-3-\s.south) -- (\t-1-1.north);
          \end{tikzpicture}

          \texttt{11: }$G' - d$ ist vollständig. $(G' - d) = (\{b,e,i\},\{\{b,e\},\{b,i\},\{e,i\}\})$.  \\

          \begin{figure}[H]
            \begin{tikzpicture}
              \draw (4,4) node[draw, circle, inner sep=1.5pt]                   (b) {b};
              \draw (5,1) node[draw, circle, inner sep=1.5pt]                  (e) {e};
              \draw (4,0) node[draw, circle, inner sep=2pt]                  (i) {i};
              \draw (b) -- (e);  
              \draw (e) -- (i);
              \draw (i) -- (b);

  
              \end{tikzpicture}
            \end{figure}
            Linked List bleibt gleich.

            \texttt{11: }$V - b$ ist nicht vollständig. \\

            \begin{figure}[H]
            \centering
          \begin{tikzpicture}
            \draw (2.5,4) node[draw, circle, inner sep=1.5pt]               (a) {a};
            \draw (1,2) node[draw, circle, inner sep=1.5pt]                  (c) {c};
            \draw (3,1) node[draw, circle, inner sep=1.5pt]                  (d) {d};
            \draw (5,1) node[draw, circle, inner sep=1.5pt]                  (e) {e};
            \draw (2,0) node[draw, circle, inner sep=1.5pt]                  (h) {h};
            \draw (4,0) node[draw, circle, inner sep=2pt]                  (i) {i};
            \draw (c) -- (a);
            \draw (c) -- (e);
            \draw (c) -- (d);
            \draw (c) -- (h);
            \draw (h) -- (d);
            \draw (h) -- (a);
            \draw (h) -- (i);
            \draw (i) -- (a);
            \draw (i) -- (e);
            \draw (e) -- (a);
            \draw (d) -- (e);
            \draw (d) -- (a);
            \draw (h) -- (e);

            \node[text width=2cm] at (3.8,-1) 
              {$V - b$};

            \end{tikzpicture}
          \end{figure}

          \texttt{12,8,9: }Knoten mit minimalen Knotengrad ist $i$, also $\alpha := i$. \\
          $G' = (\{a,e,i,h\},\{\{a,e\},\{a,i\},\{a,h\},\{e,i\},\{e,h\},\{i,h\}\})$ 

          \begin{figure}[H]
          \begin{tikzpicture}
            \draw (2.5,4) node[draw, circle, inner sep=1.5pt]               (a) {a};{d};
            \draw (5,1) node[draw, circle, inner sep=1.5pt]                  (e) {e};
            \draw (2,0) node[draw, circle, inner sep=1.5pt]                  (h) {h};
            \draw (4,0) node[draw, circle, inner sep=2pt]                  (i) {i};
            \draw (e) -- (a);
            \draw (i) -- (a);
            \draw (h) -- (a);
            \draw (e) -- (i);
            \draw (i) -- (h);
            \draw (e) -- (h);
  
            \end{tikzpicture}
          \end{figure}

          \texttt{10,2-6: }$G'$ ist vollständig. \\
         \begin{tikzpicture}
            \matrix[table] (A)
            { |[draw=none]|
                  &a&b&c&d&e&f&g&h&i\\
              maxC & 4 & 4 & 2 & 3 & 4 & 4 & 2 & 4 & 4 \\
              maxC & & & & & & & & & \\
            };
            \matrix[table,below=of A-3-2] (B) {c\\g\\};
            \matrix[table,below=of A-3-6] (C) {b\\e\\f\\i\\};
            \matrix[table,below=of A-3-4] (D) {b\\d\\e\\};
            \matrix[table,below=of A-3-8] (E) {a\\e\\i\\h\\};
            \foreach \s/\t in {8/B,4/B,3/C,6/C,7/C,10/C,5/D,2/E,9/E}
              \draw[-stealth',shorten >=3pt,shorten <=3pt] (A-3-\s.south) -- (\t-1-1.north);
          \end{tikzpicture}

          \texttt{11: }$V - i$ ist vollständig. \\

          \begin{figure}[H]
            \centering
          \begin{tikzpicture}
            \draw (2.5,4) node[draw, circle, inner sep=1.5pt]               (a) {a};
            \draw (1,2) node[draw, circle, inner sep=1.5pt]                  (c) {c};
            \draw (3,1) node[draw, circle, inner sep=1.5pt]                  (d) {d};
            \draw (5,1) node[draw, circle, inner sep=1.5pt]                  (e) {e};
            \draw (2,0) node[draw, circle, inner sep=1.5pt]                  (h) {h};

            \draw (c) -- (a);
            \draw (c) -- (e);
            \draw (c) -- (d);
            \draw (c) -- (h);
            \draw (h) -- (d);
            \draw (h) -- (a);
            \draw (e) -- (a);
            \draw (d) -- (e);
            \draw (d) -- (a);
            \draw (h) -- (e);

            \node[text width=2cm] at (3.8,-1) 
              {$V - i$};

            \end{tikzpicture}
          \end{figure}

          \begin{tikzpicture}
            \matrix[table] (A)
            { |[draw=none]|
                  &a&b&c&d&e&f&g&h&i\\
              maxC & 4 & 4 & 2 & 3 & 4 & 4 & 2 & 4 & 4 \\
              maxC & & & & & & & & & \\
            };
            \matrix[table,below=of A-3-2] (B) {c\\g\\};
            \matrix[table,below=of A-3-6] (C) {b\\e\\f\\i\\};
            \matrix[table,below=of A-3-4] (D) {b\\d\\e\\};
            \matrix[table,below=of A-3-8] (E) {a\\e\\i\\h\\};
            \foreach \s/\t in {8/B,4/B,3/C,6/C,7/C,10/C,5/D,2/E,9/E}
              \draw[-stealth',shorten >=3pt,shorten <=3pt] (A-3-\s.south) -- (\t-1-1.north);
          \end{tikzpicture}

          \textbf{\textcolor{red}{Problem bei Codezeile \texttt{4:}}} \\
          \textbf{\texttt{if (|V| - 1 > max C[v])}} \\
          \texttt{C[a] = 4, |V| - 1 = 4, \textbf{4 $\ngtr 4$}} \\

          Die If Bedingung ist hier nicht erfüllt, somit würde der Algorithmus die Zeilen \texttt{4-5} überspringen. Der Autor fürt jedoch in diesem Beispiel diese Zeilen aus. \\
          Annahme: Codezeile \texttt{4} müsste folgendermaßen geändert werden:\\
          \textbf{\texttt{if (|V| > max C[v])}} \\
          Man erhält in beiden Fällen die größte Clique mit $5$ Knoten.
          Die Liste bzw. verlinkte Liste ist jedoch eine andere. Diese ist nicht korrekt. 
          
          \textbf{Ergebnis \textcolor{red}{ohne} Änderung:}

          \begin{tikzpicture}
            \matrix[table] (A)
            { |[draw=none]|
                  &a&b&c&d&e&f&g&h&i\\
              maxC & \textcolor{red}{4} & \textcolor{ForestGreen}{4} & \textcolor{ForestGreen}{5} & \textcolor{ForestGreen}{5} & \textcolor{red}{4} & \textcolor{ForestGreen}{4} & \textcolor{ForestGreen}{2} & \textcolor{red}{4} & \textcolor{ForestGreen}{4} \\
              maxC & & & & & & & & & \\
            };
            \matrix[table,below=of A-3-2] (B) {c\\g\\};
            \matrix[table,below=of A-3-6] (C) {b\\e\\f\\i\\};
            \matrix[table,below=of A-3-4] (D) {a\\c\\d\\e\\h\\};
            \matrix[table,below=of A-3-8, fill=red!40 ] (E) {a\\e\\i\\h\\};
            \foreach \s/\t in {8/B,4/D,3/C,6/C,7/C,10/C,5/D,2/E,9/E}
              \draw[-stealth',shorten >=3pt,shorten <=3pt] (A-3-\s.south) -- (\t-1-1.north);
            \draw (5,-1) -- (7,-5)

              \node[text width=4cm] at (7,0) 
              {Maximale Clique enthält $5$ Knoten: $\{a,c,d,e,h\}$};
          \end{tikzpicture}

          \textbf{Ergebnis \textcolor{red}{mit} Änderung:}

          \begin{tikzpicture}
            \matrix[table] (A)
            { |[draw=none]|
                  &a&b&c&d&e&f&g&h&i\\
              maxC & \textcolor{ForestGreen}{5} & \textcolor{ForestGreen}{4} & \textcolor{ForestGreen}{5} & \textcolor{ForestGreen}{5} & \textcolor{ForestGreen}{5} & \textcolor{ForestGreen}{4} & \textcolor{ForestGreen}{2} & \textcolor{ForestGreen}{5} & \textcolor{ForestGreen}{4} \\
              maxC & & & & & & & & & \\
            };
            \matrix[table,below=of A-3-2] (B) {c\\g\\};
            \matrix[table,below=of A-3-6] (C) {b\\e\\f\\i\\};
            \matrix[table,below=of A-3-4] (D) {a\\c\\d\\e\\h\\};
            \foreach \s/\t in {8/B,4/D,3/C,6/D,7/C,10/C,5/D,2/D,9/D}
              \draw[-stealth',shorten >=3pt,shorten <=3pt] (A-3-\s.south) -- (\t-1-1.north);

              \node[text width=4cm] at (7,0) 
              {Maximale Clique enthält $5$ Knoten: $\{a,c,d,e,h\}$};
          \end{tikzpicture}

\end{document}
